\begin{enumerate}[label=\thesubsection.\arabic*.,ref=\thesubsection.\theenumi]
\numberwithin{equation}{enumi}
\item
Complete the table entries given below :\\


\solution  

\begin{table}[!ht]
\centering
\input{./tables/es17btech11019/initial_table.tex}
\caption{Initial table}
\label{Table}
\end{table}

\begin{align}

 We have to fill this table from part a to part g. We can neglect r_{\pi}    

\end{align}



\item
We have to find the missing values of the table.\\

\solution
We will solve it part by part.\\


\\

The Full Circuit referred to this problem is shown below which is a CC-CB amplifier : 



Below is the circuit :

\vspace{100cm}

\begin{figure}[!ht]
	\begin{center}
				\resizebox{\columnwidth}{!}{
 \begin{circuitikz}[american resistors]
  
  \ctikzset{bipoles/length=1cm}
  
  \draw[color=black]   
    (0,0) node[nigbt] (dpair1) {}
    (2,0) node[nigbt, xscale=-1] (dpair2) {}
    ($(dpair1.S)!0.5!(dpair2.S)$) node [] (midsource) {}
    (dpair1.S) to[short,-*,label=$E$] (midsource) to[short,-] (dpair2.S)
    (midsource) to[cisource, l= $2I$] (1,-2) to[short,-*,label=$-V_{EE}$](1,-2.1){}
    (dpair2.D) to[short,o-,label=$C_2$](2,1) to[cisource,l=$I$](2,2)

    to[short,-*,label=$V_{cc}$](2,3)
    (2,1) to[short,-o,label= $V_o$](3,1)
    (dpair1.D) to[short,-*,label=$C_1$](0,1)
    (dpair1.G) to[short,-,label=$B_1$](-0.75,0)
    (-0.75,0) to[short,-o,label=$V_i$](-1.5,0)
    (dpair2.G) to[short,-o,label=$B_2$](3,0)
    (3,0) to[short] node[ground] {} (3,-1)
    
  ;
 
 
\end{circuitikz}}
	\end{center}
\caption{Complete Circuit }
\label{fig:circuit_1}
\end{figure}


\item

First we will represent the given circuit using a Small Signal Equivalent Model.\\

\solution The simplified small signal circuit for the above complete circuit is shown below :

\begin{figure}[!ht]
	\begin{center}
				\resizebox{\columnwidth}{!}{\begin{circuitikz}
\ctikzset{bipoles/length=1cm}

\draw (0,-0.2)--(0,-0.2)node[ground]{};
\draw node at (0,0.8){$+$};
\draw node at (0,0){$-$};
\draw node at (0,0.4){$V_{sig}$};
\draw (0,1)--(0.5,1) to[R,l_=$R_{sig}$,-](1.5,1)to(2,1);
\draw (2,1) --(2,0.5)to[R,l_=$r_\pi$,-](2,-0.5)--(2,-1);
\draw (0,1)to[short,-o](0,1);
\draw (0,-0.2)to[short,-o](0,-0.2);
\draw (2,1)--(3.5,1)--(3.5,0.30)to[C,l_=$C_\pi$,-](3.5,-0.3)--(3.5,-1)--(3.5,-1.7)to[C,l_=$C_\pi$,-](3.5,-2.3)--(3.5,-3);
\draw (2,-1)--(3.5,-1);
\draw (2,-1) --(2,-1.5)to[R,l_=$r_\pi$,-](2,-2.5) --(2,-3);
\draw (2,-3)--(3.5,-3);
\draw --(3.5,1)--(4.2,1)to[C,l_=$C_\mu$,-](4.8,1)--(5.5,1);
\draw (3.5,-1)--(5.5,-1);
\draw (3.5,-3)--(4.2,-3)to[C,l_=$C_\mu$,-](4.8,-3)--(5.5,-3);
\draw (5.5,1)to[cisource, l= $g_m V_{\pi 1}$](5.5,-1);
\draw (5.5,-3)to[cisource, l= $g_m V_{\pi 2}$](5.5,-1);
\draw (5.5,1)--(6.5,1)node[ground]{};
\draw (5.5,-3)--(6,-3)to[R,l_=$R_{L}$,-](7,-3)--(7.5,-3)node[ground]{};
\draw (2,-3)--(0,-3)node[ground]{};
\draw
node at (2.2,0.8){$+$}
node at (2.2,-0.8){$-$}
node at (2.3,0){$V_{\pi 1}$}
node at (2.2,-1.2){$-$}
node at (2.2,-2.8){$+$}
node at (2.3,-2){$V_{\pi 2}$}
node at (2,1.3){$B_1$}
node at (5.5,1.3){$C_1$}
node at (2,-3.3){$B_2$}
node at (5.5,-3.3){$C_2$}
node at (5.8,-1){$E$}
node at (6.3,-4.5){$V_{o}$}
;

\draw (5.2,-3)--(5.2,-4.5)--(6,-4.5)to[short,-o](6,-4.5);
\end{circuitikz}}
	\end{center}
\caption{Small signal Model }
\label{fig:circuit_1}
\end{figure}


\item
Part A : We have to find values of :
 $r_{e}$ , $g_{m}$ , $r_{\pi}$ ,$C_{\pi}$ , $f_{\beta}$
 
\\

\solution

\begin{align}
   
   r_{e}  = \cfrac{V_{T}}{I_{E}}\\ 
          
          
          = 25/2  
          
          
          = 12.5\ohm \\
        
  
  r_{e} = 12.5 \ohm
    
\end{align}

\begin{align}

g_{m}   = \cfrac{I_{C}}{V_{T}}\\ 
        
        = (\cfrac{\beta}{\beta+1})\cfrac{I_{E}}{V_{T}}\\
    
        = (\cfrac{100}{100+1})\cfrac{2}{25}\\
        
        
        
        g_{m}=79.2 \cfrac{mA}{V}
        
\end{align}

\begin{align}

     r_{\pi}  = \cfrac{\beta}{g_{m}}\\
              
              = \cfrac{100}{79.2 * 0.001}\\
              
              = 1.26 k\ohm\\
              
               
        
     r_{\pi} = 1.26 k\ohm
     
\end{align}
        
        
\begin{align}
     
     f_{\beta} = \cfrac{f_{T}}{\beta}\\
               
               = \cfrac{500 * 10^6}{100}\\
               
               = 5 Mhz\\
              
              
        
     f_{\beta} = 5 Mhz
        

\end{align}
     
\begin{align}
     
     C_{\pi} = \cfrac{g_{m}}{2\pif_{T}}- C-{\mu}\\
             
    
              
             = \cfrac{79.2 * 0.001}{2*\pi*{10^8}}- (2*{10^-12})\\
              
           
              
              = 23 pF\\
              
    C_{\pi} = 23 pF
        
\end{align}


\item
Part B : We have to find values of :
 $I_{e}$ , $g_{m}$ , $r_{\pi}$ ,$\beta_{o}$ , $f_{T}$
 
\\
 
 \solution

\begin{align}

    I_{E} = \cfrac{V_{T}}{r_{e}}\\
          
          = \cfrac{25}{25}\\
          
          = 1 mA\\
        
    
    I_{E} = 1 mA
        
\end{align}

\begin{align}
    
    g_{m} = \cfrac{I_{E}}{V_{T}}\\
          
          = \cfrac{1}{25}\\
          
          = 40 mA/V\\
        
        
    g_{m}= 40 \frac{mA}{V}

\end{align}

\begin{align}
    
    r_{\pi} = \cfrac{1}{2\pi(C_{\pi}+C_{\mu})f_{\beta}}\\
      
            = \cfrac{1}{2\pi(10.7+2)*{10^{-12}}*4*10^6}\\
              
            = 3.13 k\ohm\\
              
              
        
    r_{\pi} = 3.13 k\ohm
        
    
\end{align}        
    
\begin{align}
     
    \beta_{o} = g_{m}r_{\pi}\\
      
              = 3.13*{10^3}*40*10^{-3}\\
              
              = 125 \\
        
    
    \beta_{o} = 125 
        
\end{align}        
 
\begin{align}
 
    f_{T} = \beta f_{\beta}\\
          
          = 125 * 4 * 10^6\\
             
          = 500 MHz\\

    f_{T} = 500 MHz  
     
\end{align}       

\item
Part C : We have to find values of :
 $f_{\beta}$ , $g_{m}$ , $r_{e}$ ,$I_{E}$ , $C_{\mu}$\\
 
\solution

\begin{align}

     f_{\beta} = \cfrac{{T}}{\beta} \\
               
               = \cfrac{500*{10^6}}{100} \\
        
               = 5 MHz \\
        
    f_{\beta} = 5 MHz
\end{align} 

\begin{align}
    
    g_{m} = \cfrac{\beta}{r_{\pi}} \\
    
          = \cfrac{100}{2500} \\
        
          = 40 mA/V \\
        
    g_{m}= 40 \frac{mA}{V}
        
\end{align}

\begin{align}

      I_{E} = g_{m}V_{T}\\
            
            = 40 * 25\\
            
            = 1 mA\\
            
      I_{E} = 1 mA
 
\end{align}
     
\begin{align}
    
      r_{e} = \cfrac{V_{T}}{I_{E}}\\
         
            = \cfrac{25}{1}\\
           
            = 25 \ohm\\
              
      r_{e} = 25 \ohm 
\end{align}

\begin{align}

      C_{\mu} = \cfrac{g_{m}}{2\pif_{T}}- C-{\pi}\\
            
              = \cfrac{40 * 0.001}{2*\pi*500*{10^8}}- (10.7*{10^{-12}})\\
              
              = 2.03 pF\\
        
      C_{\mu} = 2.03 pF 
    
\end{align}
    


\item
Part D : We have to find values of :
 $f_{\beta}$ , $g_{m}$ , $r_{e}$ ,$r_{\pi}$ , $C_{\pi}$
 
\\
  
 \solution

\begin{align}

        r_{e} = \cfrac{V_{T}}{I_{E}}\\
              
              = \cfrac{25}{10}\\
              
              = 2.5 \ohm\\
 
        r_{e} = 2.5 \ohm
      
\end{align}

\begin{align}

       g_{m} = \cfrac{I_{E}}{V_{T}}\\
            
            = \cfrac{10}{25}\\
            
            = 0.4 A/V\\
        
        g_{m}= 0.4 \cfrac{A}{V}
    
\end{align}

\begin{align}

        r_{\pi} = \cfrac{\beta}{g_{m}}\\
                
                = \cfrac{100}{0.4}\\
                
                = 250\ohm\\
            
        r_{\pi} = 250\ohm
    
\end{align}     
         
\begin{align}
     
        f_{\beta} = \cfrac{{T}}{\beta}\\
                  
                  = \cfrac{500*{10^6}}{100}\\
                  
                  = 5 MHz\\
        
        f_{\beta} = 5 MHz

\end{align}
 
\begin{align}

         C_{\pi} = \cfrac{g_{m}}{2\pi f_{T}}- C_{\mu}\\
                 
                 =\cfrac{0.4}{2*\pi*500*{10^6}}- (2*{10^{-12}})\\
              
                 = 125 pF\\
              
         C_{\pi} = 125 pF 

\end{align} 
    


\item
Part E : We have to find values of :
 $f_{\beta}$ , $g_{m}$ , $r_{e}$ ,$r_{\pi}$ , $C_{\pi}$ \\
  
\solution

\begin{align}

        r_{e} = \cfrac{V_{T}}{I_{E}}\\
              
              = \cfrac{25}{0.1}\\
              
              = 250 \ohm \\
              
        r_{e} = 250 \ohm
        
\end{align}

\begin{align}
    
        g_{m} = \cfrac{I_{E}}{V_{T}}\\
              
              = \cfrac{0.1}{25}\\
              
              =  4 mA/V\\
        
        g_{m}=  4 \frac{mA}{V}

\end{align}
 
\begin{align}

        r_{\pi} = \cfrac{\beta}{g_{m}}\\
                
                = \cfrac{100}{0.004}\\
                
                = 25 K\ohm\\
            
        r_{\pi} = 25 K\ohm
    
\end{align}    

\begin{align} 

        f_{\beta} = \cfrac{{T}}{\beta} \\
                  
                  = \cfrac{150*{10^6}}{100} \\
                  
                  = 1.5 MHz \\
        
       f_{\beta} = 1.5 MHz
\end{align}

\begin{align}

         C_{\pi} = \cfrac{g_{m}}{2\pi f_{T}}- C_{\mu}\\
                 
                 = \cfrac{0.004}{2*\pi*150*{10^6}}- (2*{10^{-12}}) \\
              
                 = 2.24 pF \\
    
          C_{\pi} = 2.24 pF 

\end{align}


\item
Part F : We have to find values of :
 $f_{\beta}$ , $g_{m}$ , $r_{e}$ ,$r_{\pi}$ , $C_{\pi}$ \\
  
 \solution

\begin{align}

        r_{e} = \cfrac{V_{T}}{I_{E}}\\
              
              = \cfrac{25}{1}\\
              
              = 25 \ohm\\
              
        r_{e} = 25 \ohm
                   
\end{align}

\begin{align}
    
       g_{m} = \cfrac{I_{E}}{V_{T}}\\
             
             = \cfrac{1}{25}\\
             
             =  40 mA/V\\
        
        g_{m}=  40 \cfrac{mA}{V}
        
\end{align}

\begin{align}
          
        r_{\pi} = \cfrac{\beta}{g_{m}}\\
                
                = \cfrac{100}{0.004}\\
                
                = 2500 \ohm\\
            
         r_{\pi} = 2500 \ohm
  
\end{align}

\begin{align}

        f_{\beta} = \cfrac{{T}}{\beta}\\
                  
                  = \cfrac{500*{10^6}}{10}\\
                  
                  = 50 MHz\\

       f_{\beta} = 50 MHz
\end{align}

\begin{align}

        C_{\pi} = \cfrac{g_{m}}{2\pi f_{T}}- C_{\mu}\\
              
                = \cfrac{0.04}{2*\pi*500*{10^6}}- (2*{10^{-12}})\\
              
                = 10.7 pF\\
        
        C_{\pi} = 10.7 pF

\end{align}

Part G : We have to find values of :
 $\beta$,$f_{\beta}$ , $g_{m}$ , $r_{e}$ ,$I_{E}$\\ 
  
 
\solution

\begin{align}

        \beta = \cfrac{{T}}{f_{\beta}}\\
             
              = \cfrac{800}{80}\\
              
              = 10\\
        
        \beta = 10

\end{align}
 
 \begin{align}  
        
        r_{\pi} = \cfrac{1}{2\pi(C_{\pi}+C_{\mu})f_{\beta}}\\
    
                = \cfrac{1}{2\pi(9+1)*{10^{-12}}*80*10^6}\\
              
                = 199 \ohm\\
        
        r_{\pi} = 199 \ohm
 
\end{align}

\begin{align}
   
       g_{m} = \cfrac{\beta}{r_{\pi}}\\
            
             = \frac{10}{199}\\
             
             = 50 mA/V\\
        
        g_{m}= 50 \cfrac{mA}{V}
        
\end{align}

\begin{align}        
               
        I_{E} = g_{m}V_{T}\\
              
              = 0.05 * 0.025\\
              
              = 1.25 mA\\
            
        I_{E} = 1.25 mA

\end{align}     
     
\begin{align}
    
        r_{e} = \cfrac{V_{T}}{I_{E}}\\
              
              = \cfrac{25}{1.25}\\
              
              = 20 \ohm\\
 
        r_{e} = 20 \ohm
        
\end{align}

\item
Given the following values the final table is : 
\begin{table}[!ht]
\centering
\input{.tables/es17btech11019/final_table.tex}
\caption{}
\label{table: Output_Table}
\end{table}


\item
Verify the above calculations using a Python code.\\

\solution
\begin{lstlisting}
codes/es17btech11019/es17btech11019_calc.py
\end{lstlisting}

\end{enumerate}