%%%%%%%%%%%%%%%%%%%%%%%%%%%%%%%%%%%%%%%%%
% Beamer Presentation
% LaTeX Template
% Version 1.0 (10/11/12)
%
% This template has been downloaded from:
% http://www.LaTeXTemplates.com
%
% License:
% CC BY-NC-SA 3.0 (http://creativecommons.org/licenses/by-nc-sa/3.0/)
%
%%%%%%%%%%%%%%%%%%%%%%%%%%%%%%%%%%%%%%%%%

%----------------------------------------------------------------------------------------
%	PACKAGES AND THEMES
%----------------------------------------------------------------------------------------

\documentclass{beamer}

\mode<presentation> {

% The Beamer class comes with a number of default slide themes
% which change the colors and layouts of slides. Below this is a list
% of all the themes, uncomment each in turn to see what they look like.

%\usetheme{default}
%\usetheme{AnnArbor}
%\usetheme{Antibes}
%\usetheme{Bergen}
%\usetheme{Berkeley}
%\usetheme{Berlin}
%\usetheme{Boadilla}
%\usetheme{CambridgeUS}
%\usetheme{Copenhagen}
%\usetheme{Darmstadt}
%\usetheme{Dresden}
%\usetheme{Frankfurt}
%\usetheme{Goettingen}
%\usetheme{Hannover}
%\usetheme{Ilmenau}
%\usetheme{JuanLesPins}
%\usetheme{Luebeck}
\usetheme{Madrid}
%\usetheme{Malmoe}
%\usetheme{Marburg}
%\usetheme{Montpellier}
%\usetheme{PaloAlto}
%\usetheme{Pittsburgh}
%\usetheme{Rochester}
%\usetheme{Singapore}
%\usetheme{Szeged}
%\usetheme{Warsaw}

% As well as themes, the Beamer class has a number of color themes
% for any slide theme. Uncomment each of these in turn to see how it
% changes the colors of your current slide theme.

%\usecolortheme{albatross}
%\usecolortheme{beaver}
%\usecolortheme{beetle}
%\usecolortheme{crane}
%\usecolortheme{dolphin}
%\usecolortheme{dove}
%\usecolortheme{fly}
%\usecolortheme{lily}
%\usecolortheme{orchid}
%\usecolortheme{rose}
%\usecolortheme{seagull}
%\usecolortheme{seahorse}
%\usecolortheme{whale}
%\usecolortheme{wolverine}

%\setbeamertemplate{footline} % To remove the footer line in all slides uncomment this line
\setbeamertemplate{footline}[page number] % To replace the footer line in all slides with a simple slide count uncomment this line

\setbeamertemplate{navigation symbols}{} % To remove the navigation symbols from the bottom of all slides uncomment this line
}

\usepackage{graphicx} % Allows including images
\usepackage{booktabs} % Allows the use of \toprule, \midrule and \bottomrule in tables
%\usepackage {tikz}
\usepackage{tkz-graph}
\usepackage{amsmath}
\GraphInit[vstyle = Shade]
\tikzset{
  LabelStyle/.style = { rectangle, rounded corners, draw,
                        minimum width = 2em, fill = yellow!50,
                        text = red, font = \bfseries },
  VertexStyle/.append style = { inner sep=5pt,
                                font = \normalsize\bfseries},
  EdgeStyle/.append style = {->, bend left} }
\usetikzlibrary {positioning}
%\usepackage {xcolor}
\definecolor {processblue}{cmyk}{0.96,0,0,0}
%----------------------------------------------------------------------------------------
%	TITLE PAGE
%----------------------------------------------------------------------------------------

\title[Short title]{Problem : To find the cross-over frequency of the  given transfer function} % The short title appears at the bottom of every slide, the full title is only on the title page

\author{SAHIL KUMAR SINGH} % Your name
\institute % Your institution as it will appear on the bottom of every slide, may be shorthand to save space
{
CONTROL SYSTEM\\ % Your institution for the title page
\medskip
}
\date{ES17BTECH11019} % Date, can be changed to a custom date

\begin{document}

\begin{frame}
\titlepage % Print the title page as the first slide
\end{frame}



%----------------------------------------------------------------------------------------
%	PRESENTATION SLIDES
%----------------------------------------------------------------------------------------

%------------------------------------------------


\begin{frame}{What is phase crossover frequency}
    \begin{itemize}
        \item The phase crossover is a frequency at which phase angle first reaches -180 degree or at which frequency the imaginary part of denominator of transfer function is equal to zero.
    \end{itemize}
\end{frame}



\begin{frame}{Given transfer function}
    \begin{itemize}
    
    \item G(s)=\cfrac{100}{(s+1)^3}
    
\item Now we will put s=j\omega 

\end{itemize}
\item
\begin{itemize}
\newline 
Phase crossover frequency is when phase is 180 degree that is the nyquist plot crosses the real axis i.e We will equate the imaginary part of denominator to zero and that particular frequency is the phase crossover frequency. 
        
    \end{itemize}
\end{frame}


\begin{frame}{solution}
    
      G(j\omega)=\cfrac{100}{(j\omega+1)^3}
     
      = \cfrac{100}{(j\omega)^3+1+3(j\omega)^2+3j\omega}\\
     
      = \cfrac{100}{-j\omega^3+1-3\omega^2+3j\omega}\\
     
      = \cfrac{100}{(1-3\omega^2)+j(3\omega-\omega^3)}\\
     \vspace{1cm}
     Now equating the imaginary part to zero
    
\end{frame}




\begin{frame}{Continued}
    
        (3\omega-\omega^3)=0\\
        
        
         \omega(3-\omega^2)=0\\
        
         \omega = \sqrt{3}
         
         as phase should be greater than zero for stability.
        
            
    
\end{frame}

\begin{frame}{Nyquist plot }
\begin{center}
    \includegraphics[scale = 0.3]{nyquist2}
\end{center}


    
\end{frame}


\end{document}


